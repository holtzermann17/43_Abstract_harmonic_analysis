\documentclass[12pt]{article}
\usepackage{pmmeta}
\pmcanonicalname{TopologicalGroupRepresentation}
\pmcreated{2013-03-22 18:02:18}
\pmmodified{2013-03-22 18:02:18}
\pmowner{asteroid}{17536}
\pmmodifier{asteroid}{17536}
\pmtitle{topological group representation}
\pmrecord{8}{40559}
\pmprivacy{1}
\pmauthor{asteroid}{17536}
\pmtype{Definition}
\pmcomment{trigger rebuild}
\pmclassification{msc}{43A65}
\pmclassification{msc}{22A25}
\pmclassification{msc}{22A05}
\pmsynonym{representation of topological groups}{TopologicalGroupRepresentation}
\pmdefines{equivalent representations of topological groups}

% this is the default PlanetMath preamble.  as your knowledge
% of TeX increases, you will probably want to edit this, but
% it should be fine as is for beginners.

% almost certainly you want these
\usepackage{amssymb}
\usepackage{amsmath}
\usepackage{amsfonts}

% used for TeXing text within eps files
%\usepackage{psfrag}
% need this for including graphics (\includegraphics)
%\usepackage{graphicx}
% for neatly defining theorems and propositions
%\usepackage{amsthm}
% making logically defined graphics
%%%\usepackage{xypic}

% there are many more packages, add them here as you need them

% define commands here

\begin{document}
\section{Finite Dimensional Representations}

Let $G$ be a topological group and $V$ a finite-dimensional normed vector space. We denote by $GL(V)$ the general linear group of $V$, endowed with the topology coming from the operator norm.

Regarding only the group structure of $G$, recall that a representation of $G$ in $V$ is a group homomorphism $\pi:G \longrightarrow GL(V)$.

$\,$

{\bf Definition -} A {\bf representation} of the topological group $G$ in $V$ is a continuous group homomorphism $\pi:G \longrightarrow GL(V)$, i.e. is a continuous representation of the abstract group $G$ in $V$.

$\,$

We have the following equivalent definitions:

\begin{itemize}
\item A representation of $G$ in $V$ is a group homomorphism $\pi:G \longrightarrow GL(V)$ such that the mapping $G \times V \longrightarrow V$ defined by
$(g,v) \mapsto \pi(g)v\;$
is continuous.
\end{itemize}
\begin{itemize}
\item A representation of $G$ in $V$ is a group homomorphism $\pi:G \longrightarrow GL(V)$ such that, for every $v \in V$, the mapping $G \longrightarrow V$ defined by
$g \mapsto \pi(g)v\;$
is continuous.
\end{itemize}

\section{Representations in Hilbert Spaces}
Let $G$ be a topological group and $H$ a Hilbert space. We denote by $B(H)$ the algebra of bounded operators endowed with the strong operator topology (this topology does not coincide with the norm topology unless $H$ is finite-dimensional).  Let $\mathcal{G}(H)$ the set of invertible operators in $B(H)$ endowed with the subspace topology.

$\,$

{\bf Definition -} A {\bf representation} of the topological group $G$ in $H$ is a continuous group homomorphism $\pi:G \longrightarrow \mathcal{G}(H)$, i.e. is a continuous representation of the abstract group $G$ in $H$.

$\,$

We denote by $rep(G, H)$ the set of all representations of $G$ in the Hilbert space $H$.

We have the following equivalent definitions:

\begin{itemize}
\item A representation of $G$ in $H$ is a group homomorphism $\pi:G \longrightarrow \mathcal{G}(H)$ such that the mapping $G \times H \longrightarrow H$ defined by
$(g,v) \mapsto \pi(g)v\;$
is continuous.
\end{itemize}
\begin{itemize}
\item A representation of $G$ in $H$ is a group homomorphism $\pi:G \longrightarrow \mathcal{G}(H)$ such that, for every $v \in H$, the mapping $G \longrightarrow H$ defined by
$g \mapsto \pi(g)v\;$
is continuous.
\end{itemize}

{\bf Remark -} The 3rd definition is exactly the same as the 1st definition, just written in other \PMlinkescapetext{words}.

\section{Representations as G-modules}

Recall that, for an abstract group $G$, it is the same to consider a representation of $G$ or to consider a \PMlinkname{$G$-module}{GModule}, i.e. to each representation of $G$ corresponds a $G$-module and vice-versa.

For a topological group $G$, representations of $G$ satisfy some continuity \PMlinkescapetext{properties}. Thus, we are not interested in all $G$-modules, but rather in those which are compatible with the continuity conditions.

$\,$

{\bf Definition -} Let $G$ be a topological group. A {\bf $G$-module} is a normed vector space (or a Hilbert space) $V$ where $G$ acts continuously, i.e. there is a continuous action $\psi :G \times V \longrightarrow V$.

$\,$

To give a representation of a topological group $G$ is the same as giving a $G$-module (in the sense described above).

\section{Special Kinds of Representations}

\begin{itemize}
\item \item Let $\pi \in rep(G,H)$. We say that a subspace $V \subseteq H$ is {\bf \PMlinkescapetext{invariant}} by $\pi$ if $V$ is invariant under every operator $\pi(s)$ with $s \in G$.
\end{itemize}
\begin{itemize}
\item \item A {\bf \PMlinkescapetext{subrepresentation}} of a representation $\pi \in rep(G,H)$ is a representation $\pi_0 \in rep(G,H_0)$ obtained from $\pi$ by restricting to a closed \PMlinkescapetext{invariant} subspace $H_0 \subseteq H$.
\end{itemize}
\begin{itemize}
\item \item A representation $\pi \in rep(G,H)$ is said to be {\bf \PMlinkescapetext{irreducible}} if the only closed \PMlinkescapetext{invariant} subspaces of $H$ are the trivial ones, $\{0\}$ and $H$.
\end{itemize}
\begin{itemize}
\item Two representations $\pi_1:G \longrightarrow GL(V_1)$ and $\pi_2:G \longrightarrow GL(V_2)$ of a topological group $G$ are said to be {\bf equivalent} if there exists an invertible linear transformation $T:V_1 \longrightarrow V_2$ such that for every $g \in G$ one has $\pi_1(g) = T^{-1} \pi_2(g) T$.

The definition is similar for Hilbert spaces, by taking $T$ as an invertible bounded linear operator.
\end{itemize}
%%%%%
%%%%%
\end{document}
