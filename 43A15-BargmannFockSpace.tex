\documentclass[12pt]{article}
\usepackage{pmmeta}
\pmcanonicalname{BargmannFockSpace}
\pmcreated{2013-03-22 16:42:44}
\pmmodified{2013-03-22 16:42:44}
\pmowner{ErlendA}{6587}
\pmmodifier{ErlendA}{6587}
\pmtitle{Bargmann-Fock space}
\pmrecord{14}{38928}
\pmprivacy{1}
\pmauthor{ErlendA}{6587}
\pmtype{Definition}
\pmcomment{trigger rebuild}
\pmclassification{msc}{43A15}
\pmsynonym{Fock space}{BargmannFockSpace}
\pmdefines{Fock space}

\endmetadata

% this is the default PlanetMath preamble.  as your knowledge
% of TeX increases, you will probably want to edit this, but
% it should be fine as is for beginners.

% almost certainly you want these
\usepackage{amssymb}
\usepackage{amsmath}
\usepackage{amsfonts}

% used for TeXing text within eps files
%\usepackage{psfrag}
% need this for including graphics (\includegraphics)
%\usepackage{graphicx}
% for neatly defining theorems and propositions
%\usepackage{amsthm}
% making logically defined graphics
%%%\usepackage{xypic}

% there are many more packages, add them here as you need them

% define commands here

\begin{document}
The Bargmann-Fock space (or simply Fock space) is the Hilbert space of entire functions, $\mathcal{F}^2(\mathbb{C})$ s.t. 
$$ \int_\mathbb{C}|F(z)|^2 e^{- \pi |z|^2}dx dy<\infty$$
with associated inner product
$$ \int_\mathbb{C}F(z)\overline{G(z)}e^{- \pi |z|^2}dx dy$$

where $z=x+i y$


\begin{thebibliography}{2}
\bibitem{vb} V. Bargmann, ``Remarks on a Hilbert Space of Analytic Function'' {\it Proceedings of the National Academy of Sciences of the United States of America} {\bf 48} (1962): 199 - 204
\bibitem{vbt} V. Bargmann \& I. T. Todorov, ``Spaces of analytic functions on a complex cone as carriers for the symmetric tensor representations of SO(n)'' {\it Journal of Mathematical Physics} {\bf 18} 6 (1977): 1141 - 1148
\end{thebibliography}
%%%%%
%%%%%
\end{document}
