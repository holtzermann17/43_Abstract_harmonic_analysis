\documentclass[12pt]{article}
\usepackage{pmmeta}
\pmcanonicalname{RepresentationsOfCompactGroupsAreEquivalentToUnitaryRepresentations}
\pmcreated{2013-03-22 18:02:28}
\pmmodified{2013-03-22 18:02:28}
\pmowner{asteroid}{17536}
\pmmodifier{asteroid}{17536}
\pmtitle{representations of compact groups are equivalent to unitary representations}
\pmrecord{9}{40563}
\pmprivacy{1}
\pmauthor{asteroid}{17536}
\pmtype{Theorem}
\pmcomment{trigger rebuild}
\pmclassification{msc}{43A65}
\pmclassification{msc}{22A25}
\pmclassification{msc}{22C05}
%\pmkeywords{compact group}
%\pmkeywords{unitary representation}

% this is the default PlanetMath preamble.  as your knowledge
% of TeX increases, you will probably want to edit this, but
% it should be fine as is for beginners.

% almost certainly you want these
\usepackage{amssymb}
\usepackage{amsmath}
\usepackage{amsfonts}

% used for TeXing text within eps files
%\usepackage{psfrag}
% need this for including graphics (\includegraphics)
%\usepackage{graphicx}
% for neatly defining theorems and propositions
%\usepackage{amsthm}
% making logically defined graphics
%%%\usepackage{xypic}

% there are many more packages, add them here as you need them

% define commands here

\begin{document}
\PMlinkescapephrase{representation}
\PMlinkescapephrase{representations}
\PMlinkescapephrase{equivalent}

{\bf Theorem -} Let $G$ be a compact topological group. If $(\pi, V)$ is a \PMlinkname{finite-dimensional representation}{TopologicalGroupRepresentation} of $G$ in a normed vector space $V$, then $\pi$ is \PMlinkname{equivalent}{TopologicalGroupRepresentation} to a unitary representation.

$\,$

{\bf \emph{Proof:}}  Let $\langle \cdot, \cdot \rangle$ denote an inner product in $V$. Define a new inner product in the vector space $V$ by
\begin{align*}
\langle v, w \rangle_1 := \int_G \big\langle \pi(s)v, \pi(s)w \big\rangle \;d\mu(s)
\end{align*}
where $\mu$ is a Haar measure in $G$. It is easy to see that $\langle \cdot, \cdot \rangle_1$ defines indeed an inner product, noting that $\langle \pi(\cdot)v, \pi(\cdot)w \rangle$ is a continuous function in $G$.

Now we claim that, for every $s \in G$, $\pi(s)$ is a unitary operator for this new inner product. This is true since
\begin{eqnarray*}
\big\langle \pi(t)v, \pi(t)w \big\rangle_1 & = & \int_G \big\langle \pi(s)\pi(t)v, \pi(s)\pi(t)w \big\rangle \;d\mu(s)\\
& = & \int_G \langle \pi(st)v, \pi(st)w \rangle \;d\mu(s)\\
& = & \int_G \langle \pi(s)v, \pi(s)w \rangle \;d\mu(s)\\
& = & \langle v, w \rangle_1
\end{eqnarray*}

Denote by $V'$ the space $V$ endowed the inner product $\langle \cdot, \cdot \rangle_1$. As we have seen, $(\pi, V')$ is a unitary representation of $G$ in $V'$. Of course, $(\pi, V')$ and $(\pi, V)$ are equivalent representations, since
\begin{align*}
\pi=\mathrm{id}^{-1}\,\pi\,\mathrm{id}
\end{align*}
where $\mathrm{id}:V \longrightarrow V'$ is the identity mapping. Thus, $(\pi, V)$ is equivalent to a unitary representation. $\square$
%%%%%
%%%%%
\end{document}
